\documentclass{article}
\usepackage{graphicx}
\usepackage{geometry}
\usepackage{fontspec}
%\usepackage [thaifont = Angsana New]{thaispec}

\geometry{a4paper, margin=1in}

% Set the main font to Angsana New
% \setmainfont{Angsana New}

\title{
    \includegraphics[width=0.2\textwidth]{satit ram logo.png}\\
    \fontsize{18pt}{18pt}\selectfont
    The Demonstration School of Ramkhamhaeng University English Program\\
    SCI 23101 Mathayom 3 Midterm Examination\\
    First Semester, 2024
}
\author{}
\date{}

\begin{document}

\maketitle

\begin{flushleft}
\begin{tabular}{l l l l}
Name: \underline{\hspace{5cm}} & No.: \underline{\hspace{2cm}} & Class: \underline{\hspace{2cm}} & Mark: \underline{\hspace{2cm}} /45 \\
\end{tabular}
\end{flushleft}

\section*{Multiple Choice Questions}



\begin{enumerate}
    \item Which of the following best defines a species in biological terms?
    \begin{enumerate}
        \item A group of organisms that share similar physical characteristics.
        \item A group of organisms that can interbreed and produce fertile offspring.
        \item A group of organisms that live in the same habitat.
        \item A group of organisms that have the same genetic makeup.
    \end{enumerate}

    \item What is a population in ecological terms?
    \begin{enumerate}
        \item All the different species living in a particular area.
        \item A group of individuals of the same species living in a specific area.
        \item The physical environment where organisms live.
        \item The interaction between different species in a habitat.
    \end{enumerate}

    \item What is a community in ecological terms?
    \begin{enumerate}
        \item All the members of a single species in a particular area.
        \item The non-living components of an ecosystem.
        \item The interaction between a population and its physical environment.
        \item All the different populations of different species living and interacting in a particular area.
    \end{enumerate}

    \item What is a habitat?
    \begin{enumerate}
        \item The role an organism plays in its ecosystem.
        \item The specific physical environment in which an organism lives.
        \item The interactions between different species in an ecosystem.
        \item The migration pattern of a species.
    \end{enumerate}

    \item What is an ecosystem?
    \begin{enumerate}
        \item A group of the same species living in a specific area.
        \item The non-living components of an environment.
        \item A community of living organisms interacting with their physical environment.
        \item The role and function of an organism within its community.
    \end{enumerate}

    \item Which of the following is an example of an abiotic component of an ecosystem?
    \begin{enumerate}
        \item Plants
        \item Animals
        \item Bacteria
        \item Water
    \end{enumerate}

    \item What characterizes a balanced ecosystem?
    \begin{enumerate}
        \item An equal number of producers, consumers, and decomposers.
        \item No competition among species for resources.
        \item A state where all organisms can meet their needs and populations remain relatively stable over time.
        \item The absence of any abiotic components.
    \end{enumerate}

    \item Which of the following best describes the role of a predator in an ecosystem?
    \begin{enumerate}
        \item An organism that produces its own food through photosynthesis.
        \item An organism that is hunted and eaten by other organisms.
        \item An organism that hunts and consumes other organisms for food.
        \item An organism that decomposes dead organic material.
    \end{enumerate}

    \item What is a common adaptation of prey to avoid predators?
    \begin{enumerate}
        \item Developing sharp teeth and claws.
        \item Camouflage to blend in with their surroundings.
        \item Producing their own food through photosynthesis.
        \item Hunting in packs to increase chances of catching food.
    \end{enumerate}

    \item Which of the following best describes mutualism in symbiotic relationships?
    \begin{enumerate}
        \item One organism benefits while the other is harmed.
        \item Both organisms benefit from the relationship.
        \item One organism benefits while the other is unaffected.
        \item Neither organism benefits nor is harmed.
    \end{enumerate}

    \item Which of the following is an example of commensalism?
    \begin{enumerate}
        \item A bee pollinating a flower while gathering nectar.
        \item A bird eating parasites off a rhinoceros.
        \item Barnacles attaching to a whale and benefiting from movement without harming the whale.
        \item A tapeworm living in the intestine of a host and absorbing nutrients.
    \end{enumerate}

    \item What is interspecific competition?
    \begin{enumerate}
        \item Competition between individuals of the same species for resources.
        \item Competition between individuals of different species for the same resources.
        \item Competition between organisms and their physical environment.
        \item Competition between predators and their prey.
    \end{enumerate}

    \item Which of the following is an example of competition in an ecosystem?
    \begin{enumerate}
        \item Two species of birds feeding on the same type of insects in a forest.
        \item A bee pollinating a flower while gathering nectar.
        \item A lion hunting a zebra for food.
        \item Algae and fungi forming a lichen on a rock.
    \end{enumerate}

    \item What is chemical control in pest management?
    \begin{enumerate}
        \item The use of natural predators to control pest populations.
        \item The use of physical barriers to prevent pests from reaching crops.
        \item The use of pesticides and chemicals to reduce pest populations.
        \item The use of genetically modified organisms to resist pests.
    \end{enumerate}

    \item What is biological control in pest management?
    \begin{enumerate}
        \item The application of synthetic chemicals to eliminate pests.
        \item The use of living organisms, such as predators, parasites, or pathogens, to control pest populations.
        \item The use of mechanical devices to trap pests.
        \item The use of crop rotation to minimize pest impact.
    \end{enumerate}

    \item Which of the following exemplifies a correct food chain?
    \begin{enumerate}
        \item shark → tuna → krill → phytoplankton
        \item plant → grasshopper → spider → frog → lizard → fox → hawk
        \item plant → hawk
        \item phytoplankton → krill → tuna → shark
    \end{enumerate}

    \item What type of relationship do food web and food chain diagrams represent?
    \begin{enumerate}
        \item generative
        \item feeding
        \item reproductive
        \item growth
    \end{enumerate}

    \item An animal that obtains nutrients by eating only plants is a \underline{\hspace{5cm}}.
    \begin{enumerate}
        \item Herbivore
        \item Omnivore
        \item Carnivore
        \item Decomposer
    \end{enumerate}

    \item According to the food web above, which of the following accurately depicts a food chain from the food web?
    \begin{enumerate}
        \item Phytoplankton → sand lance → puffin
        \item Zooplankton → phytoplankton → fox
        \item Krill → phytoplankton → sand lance
        \item Salmon → krill → cephalopod
    \end{enumerate}

    \item What does a pyramid of numbers represent in an ecosystem?
    \begin{enumerate}
        \item The total biomass at each trophic level.
        \item The flow of energy between trophic levels.
        \item The number of organisms at each trophic level.
        \item The amount of energy lost at each trophic level.
    \end{enumerate}

    \item Which of the following is typically true for a pyramid of numbers in a grassland ecosystem?
    \begin{enumerate}
        \item There are more secondary consumers than primary consumers.
        \item There are more producers than primary consumers.
        \item There are more tertiary consumers than secondary consumers.
        \item There are fewer producers than primary consumers.
    \end{enumerate}

    \item What is bioaccumulation?
    \begin{enumerate}
        \item The process by which pollutants accumulate in the tissues of organisms within a food chain.
        \item The process by which organisms break down organic matter to release energy.
        \item The process by which organisms produce their own food through photosynthesis.
        \item The process by which nutrients are cycled through ecosystems.
    \end{enumerate}

    \item In Mendel's experiments with pea plants, what are the units of inheritance that he identified?
    \begin{enumerate}
        \item Chromosomes
        \item Genes
        \item Alleles
        \item Proteins
    \end{enumerate}

 \end{enumerate}

\end{document}
