\documentclass{article}

\usepackage{url}
\usepackage{parskip}

\begin{document}

\noindent This is the beginning of a poem

by Edgar Allan Poe:

\begin{center}

    \emph{Annabel Lee}

\end{center}

\begin{center}

    It was many and many a year ago,\\

    In a kingdom by the sea,\\

    That a maiden there lived whom you may know\\

    By the name of Annabel Lee

\end{center}

The complete poem can be read on

\url{http://www.online-literature.com/poe/576/}\\[10mm]



\noindent Niels Bohr said: ``An expert is a person

who has made all the mistakes that can be made in

a very narrow field.''

Albert Einstein said:

\begin{quote}

    Anyone who has never made a mistake has never

    tried anything new.

\end{quote}

Errors are inevitable. So, let's be brave

trying something new.\\[10mm]

The authors of the CTAN team listed ten good reasons

for using \TeX. Among them are:

\begin{quotation}

  \TeX\ has the best output. What you end with,

  the symbols on the page, is as useable, and beautiful,

  as a nonprofessional can produce.

  \TeX\ knows typesetting. As those plain text samples

  show, TeX's has more sophisticated typographical

  algorithms such as those for making paragraphs

  and for hyphenating.

  \TeX\ is fast. On today's machines \TeX\ is very fast.

  It is easy on memory and disk space, too.

  \TeX\ is stable. It is in wide use, with a long

  history. It has been tested by millions of users,

  on demanding input.

  It will never eat your document. Never.

\end{quotation}

The original text can be found on

\url{https://www.ctan.org/what_is_tex.html}.\\[10mm]


The authors of the CTAN team listed ten good reasons

for using \TeX. Among them are:

\TeX\ has the best output. What you end with,

the symbols on the page, is as useable, and beautiful,

as a nonprofessional can produce\ldots

The original text can be found on

\url{https://www.ctan.org/what_is_tex.html}.





\end{document}
