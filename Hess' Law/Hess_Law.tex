\documentclass{article}
\usepackage[utf8]{inputenc}

\title{Hess' Law}
\author{Richard Kiddle }
\date{August 2020}

\begin{document}

\maketitle
\section{Example One}
The enthalpy changes for two reactions are shown below:

CO(g)+\frac{1}{2}$O_{2}(g)\rightarrow CO_{2}(g)\,\,\,\,\,\,\,\,\,\,\,\,\,\,\,\,\,\,\,\,\Delta H_{1}=-283\,\,kJ\,mol^{-1}

Cu(s)+\frac{1}{2}$O_{2}(g)\rightarrow CuO(s)\,\,\,\,\,\,\,\,\,\,\,\,\,\,\,\,\,\,\,\,\Delta H_{2}=-155\,\,kJ\,mol^{-1}

Using this data, calculate the enthalpy change for the following reaction:

CuO(s)+CO(g)\rightarrow Cu(s)+CO_{2}(g)\,\,\,\,\,\,\,\,\,\,\,\,\,\,\,\,\,\,\,\,\,\Delta H=\, ?\

The target equation (which we will denote (\Delta H_{T}\) can be made by adding together equation one and the reverse of equation two.

Reversing equation two gives:

$$CuO(s)\rightarrow Cu(s)+\frac{1}{2}O_{2}(g)\,\,\,\,\,\,\,\,\,\,\,\,\,\,\,\,\,\,\,\,\Delta H=\,+155\,kJ\,mol^{-1}$$

$$\Delta H_{T}=\Delta H_{1}+(-\Delta  H_{2})$$
$$\Delta H_{T}= (-283)-(-155)$$
$$\Delta H_{T}= -283 +155$$
$$\Delta H_{T}= -128\,\,kJ\,\,mol^{-1}$$

\section{Example two}

Hess’s Law can be used to calculate the enthalpy change for the formation of ethanoic acid from its elements.

2C (s)  +  2H_{2} (g)  +  O_{2} (g)  \rightarrow   CH_{3}COOH (l)

Calculate the enthalpy change for the above reaction in kJ mol-1 using the data book and the following reaction:
$$CH_{3}COOH(l)+2O_{2}\rightarrow 2CO_{2}g)+2H_{2}O(l)\,\,\,\,\,\,\,\,\,\,\,\,\,\Delta H=-836\,kJ\,mol^{-1}$$

It is not unusual to have to assemble equations to be used yourself, relying on values from the data book.

If you are methodical, you will always arrive at the correct value for your target enthalpy.

Follow these steps:
\begin{enumerate}
\item Write out the equations to be used and number them.
\item Look at the equations to see if they need to be multiplied or reversed to suit the target equation. (NB remember to change \Delta H\) accordingly)

\item (Optional) Rewrite full equations added together to check if they cancel out to the target equation.

\item Calculate \Delta H\) for your target equation. You may wish (or be asked) to express \Delta H\) as a sum of the ∆H values for your given equations.
\item The target equation from the question is:
\end{enumerate}
2C (s)  +  2H_{2} (g)  +  O_{2} (g)  \rightarrow  CH_{3}COOH (l)\,\,\,\,\,\,\,\,\,\,\,\,\,\,\,\,\,\,\Delta H_{T}=\,?

Supplementing the given equation with the equations for the enthalpies of combustion of hydrogen and carbon from the data book gives a set of three equations we can use.

C(s)+O_{2}(g)\rightarrow CO_{2}(g)\,\,\,\,\,\,\,\,\,\,\,\,\,\,\,\,\,\,\,\,\,\,\,\,\,\,\,\,\,\,\,\Delta H_{1}= -394\,kJ\,mol^{-1}

H_{2}(g)+ \frac{1}{2}O_{2}(g)\rightarrow H_{2}O (l)\,\,\,\,\,\,\,\,\,\,\,\,\,\,\,\,\,\,\,\,\,\,\,\,\Delta H_{2}= -286\,kJ\,mol^{-1}

CH_{3}COOH(l)  +  2O_{2} (g) \rightarrow   2CO_{2} (g)  +  2H_{2}O(l)\,\,\,\,\,\,\,\Delta H_{3} = -876\,kJ\,mol^{-1}


To satisfy the target equation, we must double equation one, double equation two and reverse equation three. So,

\Delta H_{T}= (2 \times \Delta H_{1}) + (2 \times \Delta H_{2}) + (-\Delta H_{3})

Rewriting the equations in full after they have been multiplied/reversed accordingly gives:
2C (s) + 2O_{2} (g) + 2H_{2} (g) + O_{2} (g) + 2CO_{2} (g)  +  2H_{2}O (l)
\rightarrow 2CO_{2} (g) + 2H_{2}O (l)  +  CH_{3}COOH (l) + 2O_{2} (g)

By cancelling species that are present on both sides, we can see that we have arrived at our target equation:
2C (s)  +  2H_{2} (g)  +  O_{2} (g)  \rightarrow   CH_{3}COOH (l)

Finally, we must use the enthalpy change values to calculate the enthalpy change for the target equation.

\Delta H_{T}=2\Delta H_{1}+2\Delta H_{2}(-\Delta H_{3})
= [2 \times  (-394)]  +  [2 \times  (-286)]  +  (876)
=  -788  - 572  +  876
=  -484\,kJ\, mol^{-1}
\end{document}
