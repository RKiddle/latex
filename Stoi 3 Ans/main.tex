\documentclass{article}
\usepackage{amsmath}
\usepackage{chemfig}

\title{Stoichiometry Problems 3: Percent Yield}
\author{Answers}
\date{}


\begin{document}

\maketitle

\section*{Problems and Answers}

\begin{enumerate}
    \item In a reaction, 10 grams of hydrogen gas (\(\text{H}_2\)) reacts with excess oxygen gas (\(\text{O}_2\)) to produce water (\(\text{H}_2\text{O}\)). If the actual yield of water is 80 grams, calculate the percent yield.
    \[
    2\text{H}_2 + \text{O}_2 \rightarrow 2\text{H}_2\text{O}
    \]
    \textbf{Answer:} Theoretical yield = 90 grams of \(\text{H}_2\text{O}\). Percent yield = 88.9\%.

    \item When 15 grams of aluminum (\(\text{Al}\)) reacts with excess chlorine gas (\(\text{Cl}_2\)), 50 grams of aluminum chloride (\(\text{AlCl}_3\)) are produced. Calculate the percent yield.
    \[
    2\text{Al} + 3\text{Cl}_2 \rightarrow 2\text{AlCl}_3
    \]
    \textbf{Answer:} Theoretical yield = 74.5 grams of \(\text{AlCl}_3\). Percent yield = 67.1\%.

    \item If 20 grams of methane (\(\text{CH}_4\)) reacts with excess oxygen gas (\(\text{O}_2\)), and the actual yield of carbon dioxide (\(\text{CO}_2\)) is 44 grams, calculate the percent yield.
    \[
    \text{CH}_4 + 2\text{O}_2 \rightarrow \text{CO}_2 + 2\text{H}_2\text{O}
    \]
    \textbf{Answer:} Theoretical yield = 55 grams of \(\text{CO}_2\). Percent yield = 80\%.

    \item In a reaction, 25 grams of iron (\(\text{Fe}\)) reacts with excess sulfur (\(\text{S}\)) to produce iron(II) sulfide (\(\text{FeS}\)). If the actual yield of iron(II) sulfide is 40 grams, calculate the percent yield.
    \[
    \text{Fe} + \text{S} \rightarrow \text{FeS}
    \]
    \textbf{Answer:} Theoretical yield = 43.9 grams of \(\text{FeS}\). Percent yield = 91.1\%.

    \item When 30 grams of calcium carbonate (\(\text{CaCO}_3\)) reacts with excess hydrochloric acid (\(\text{HCl}\)), 22 grams of carbon dioxide (\(\text{CO}_2\)) are produced. Calculate the percent yield.
    \[
    \text{CaCO}_3 + 2\text{HCl} \rightarrow \text{CaCl}_2 + \text{CO}_2 + \text{H}_2\text{O}
    \]
    \textbf{Answer:} Theoretical yield = 13.2 grams of \(\text{CO}_2\). Percent yield = 166.7\%.
\end{enumerate}

\section*{Real-Life Applications of Percent Yield}

\begin{enumerate}
    \item Explain how percent yield is important in the pharmaceutical industry when producing medications.
    \textbf{Answer:} Percent yield is crucial in the pharmaceutical industry to ensure that the production of medications is efficient and cost-effective. High percent yields mean that more of the desired product is obtained from the starting materials, reducing waste and production costs. This is important for maintaining the affordability and availability of medications.

    \item Discuss the role of percent yield in the food industry, particularly in the production of processed foods.
    \textbf{Answer:} In the food industry, percent yield is important for maximizing the efficiency of food production processes. High percent yields ensure that the maximum amount of product is obtained from raw materials, reducing waste and improving profitability. This is particularly important in the production of processed foods, where ingredients are often expensive and waste can significantly impact costs.

    \item Describe how percent yield can impact the profitability of chemical manufacturing companies.
    \textbf{Answer:} Percent yield directly affects the profitability of chemical manufacturing companies. Higher percent yields mean that more product is obtained from the same amount of raw materials, reducing costs and increasing profits. Conversely, low percent yields can lead to higher production costs and lower profitability. Therefore, optimizing percent yield is a key focus for chemical manufacturers.

    \item Explain the significance of percent yield in environmental chemistry, especially in waste management and recycling processes.
    \textbf{Answer:} In environmental chemistry, percent yield is important for minimizing waste and improving the efficiency of recycling processes. High percent yields mean that more of the desired product is recovered from waste materials, reducing the amount of waste that needs to be disposed of and improving


\end{enumerate}

\end{document}